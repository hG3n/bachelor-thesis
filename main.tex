\documentclass[pdftex,12pt,a4paper]{report}

%------------------------------------------------
%	PACKAGES
%------------------------------------------------

\usepackage{webis}

\usepackage{mwe}

\usepackage{ucs}

\usepackage[utf8x]{inputenc}

\usepackage{ngerman}

\usepackage[gen]{eurosym}

%\usepackage{hyperref}

\usepackage{url}

\usepackage{enumitem}

\usepackage{amsmath}

\usepackage{algorithm}

\usepackage{algpseudocode}

\usepackage{array}

\usepackage{mathtools}

\usepackage[autostyle=false, style=german]{csquotes}

\usepackage{gensymb}


%------------------------------------------------
%	DEFINE AND CONFIGURE ENVIRONMENTS
%------------------------------------------------

% Figure (Latex definition)
\usepackage[font=small,labelfont=bf,width=12cm]{caption}
\usepackage{chngcntr}
\counterwithout{figure}{chapter} 

% Anforderung (Own definition)
\newtheorem{anforderung}{Anforderung}[section]
\counterwithout{anforderung}{section}

% Hypothese (Own definition)
\newtheorem{hypothese}{Hypothese}[section]
\counterwithout{hypothese}{section}


%------------------------------------------------
%	CONFIGURATION FOR COVER PAGE
%------------------------------------------------

\global\arbeit{Bachelorarbeit}

\global\titel{Stereo-basierte Echtzeit-Hinderniserkennung für unbemannte Flugsysteme}

\global\subtitel{}

\global\bearbeiter{Hagen Hiller}

\global\betreuer{Dr. Jens Kersten}

\global\erstgutachter{Prof. Dr. Volker Rodehorst}

\global\zweitgutachter{Junior-Prof. Dr. Florian Echtler}

\global\abgabetermin{22.02.2016}

\global\ort{Weimar}

\global\matrikelnummer{110514}

\global\geburtsdatum{04.06.1992}

\global\geburtsort{Berlin}

%------------------------------------------------
%	START DOCUMENT
%------------------------------------------------

\begin{document}

%------------------------------------------------
%	COVER PAGE
%------------------------------------------------

\deckblatt

%------------------------------------------------
%	DECLARATION FOR EXAMINATION OFFICE
%------------------------------------------------

\erklaerung

%------------------------------------------------
%	ABSTRACT
%------------------------------------------------

\begin{abstract}
Die Verwendung ferngesteuerter unbemannter Flugsysteme ist längst nicht mehr rein militärischer Natur. Auch in der Zivilgesellschaft werden diese zur Produktion von Filmen, zu Vermessungszwecken oder aber in privaten Bereichen verwendet. Jedoch werden diese privaten  Anwendungen zum Teil kritisch beäugt. Trotz dessen ist die autonome Erkundung von unbekannten Gebieten ein weit erforschter Bereich. \\

\noindent
Gerade in Arealen ohne verfügbare GPS Verbindung erfolgt die automatische Wegfindung unter Verwendung anderer Sensoren. Die Benutzung einer Stereo-Kamera-Konfiguration ist dabei der einfachste Weg optisch eine dreidimensionale Rekonstruktion der Szene zu erstellen anhand derer etwaige Hindernisse gesucht werden. Im Rahmen dieser Arbeit wurden zwei Systeme zur Erkennung von Hindernissen in Echtzeit entwickelt. Die gleichzeitige Lokalisierung und Kartographierung (SLAM - \emph{Simultaneous Localization and Mapping}), beispielsweise innerhalb eines einsturzgefährdeten Gebäudes ist somit ein mögliches Anwendungsgebiet. Dabei steht die sichere Navigation des Flugsystems im Vordergrund. Folglich ist gerade der Aspekt der Erkennung in Echtzeit, sowie der Erhalt der Hindernispositionen eine unabdingbare Anforderung. Die zu berechnenden Daten über die Position potentieller Hindernisse müssen dabei effizient zur Entwicklung einer Strategie zur Vermeidung verwendet werden können.\\
\end{abstract}


%------------------------------------------------
%	TABLE OF CONTENTS
%------------------------------------------------

\tableofcontents

%------------------------------------------------
%	THESIS CHAPTERS
%------------------------------------------------

\chapter{Einführung}
\label{chp:introduction}
\section{Motivation}
\label{sec:motivation}

Die vorliegende Arbeit ist im Forschungsgebiet Robotik angesiedelt und behandelt bildbasierte Verfahren zur Erkennung von Hindernissen für autonome Flugsysteme (UAV \footnote{Unmanned Aircraft System}). Diese Verfahren sind für die Navigation sowie für die autonome Erkundung von unbekannten oder unzugänglichen Gebieten unerlässlich. Dabei müssen nicht nur die physikalischen Eigenschaften der Drohne betrachtet werden sondern auch die Fusion verschiedenster Sensoren.\\

\noindent
Ein wichtiges Kriterium in der Entwicklung autonomer Roboter ist die Erkennung von Hindernissen in Echtzeit. Dabei muss jedoch zuerst definiert werden was vom System als potentielles Hindernis erkannt werden soll. Prinzipiell sind alle Objekte welche sich in der unmittelbaren Nähe des Systems befinden eine Gefahrenquelle. Im Fall eines Kamera-basierten Systems mit lediglich einer Hauptblickrichtung, ist die Detektion jedoch beschränkt, so dass eine Einschränkung der zugelassenen Manöver, z.B. auf eine Bewegung lediglich in Blickrichtung der Kamera, erfolgen sollte. Dies schließt eine Kollision mit Objekten außerhalb des Sichtfeldes aus. Auch sehr weit entfernte Objekte sind prinzipiell nicht als Hindernis anzusehen, wobei die maximal zu betrachtende Gefahrendistanz abhängig von der aktuellen Bewegungsgeschwindigkeit angepasst werden muss. Unter Betrachtung dieser Gesichtspunkte wird ein Hindernis innerhalb dieser Arbeit als ein Objekt definiert welches sich innerhalb eines definierten Distanzbereichs und innerhalb des Sichtfeldes der Kamera befindet.\\

\noindent
Die hauptsächliche Anwendung des im Rahmen dieser Arbeit entwickelten Systems zielt auf die autonome Navigation von unbemannten Flugsystemen ab. Weitere Anwendungsbereiche können sowohl komplexerer als auch einfacherer Natur sein. Prinzipiell ist es möglich die entwickelten Algorithmen und Methoden im Automobil Bereich zu verwenden um beispielsweise Objekte vor oder hinter dem Kraftfahrzeug zu erkennen und deren Distanz zu ermitteln. Weiterhin ist es möglich die aufgenommenen Punktwolken im Nachhinein zur groben dreidimensionalen Rekonstruktion der Umgebung zu verwenden. %Diese kann dann im Fall eines weiteren Fluges in der Selben Umgebung als Abgleich verwendet werden.

\section{Hardwarekomponenten}
\label{sec:setup}

% TODO Einleitung siehe Jens version
Die aktive Entwicklung der Methoden und Algorithmen erfolgte im Hinblick auf eine Verwendung dieser durch das von Ascending Technologies \cite{asctec} entwickelte UAS Pelican (Abbildung \ref{img:pelican}). Dabei handelt es sich dabei um einen Quadrocopter der speziell für Forschungszwecke entwickelt wurde. Er ist mit einem Bordcomputer ausgestattet, der die nötige Leistung für die Entwicklung der Algorithmen bereitstellt (3rd Generation Intel Core i7). Weiterhin wurden zwei MatrixVision BlueFOX mv-MLC200wC Industriekameras \cite{matrixvision} mit einem Sichtfeld von je 100$^\circ$ als visuelles System verwendet. Die maximale Auflösung beider Kameras beträgt $752\times480$ bei 60 möglichen Bildern pro Sekunde, in Abhängigkeit verschiedener Parameter (verwendetet Verschlusszeit, aufgenommene Bitrate, u. a). Für den Echtzeit-Aspekt des Systems werden beide Kameras in einem horizontalen und vertikalen Binning-Modus verwendet. Dies halbiert die Anzahl der Bildpunkte in beiden Dimensionen auf $376\times240$, wodurch der Berechnungsaufwand verringert und die die Aufnahmerate der Kameras auf bis zu 170 Einzelbilder pro Sekunde maximiert wird.

\noindent
Für die Implementierung der Methoden wurde die freie Computer Vision Bibliothek OpenCV \cite{opencv} verwendet. Diese stellt benötigte algebraische Grundoperationen sowie bestimmte Algorithmen, welche im Rahmen dieser Arbeit genutzt wurden, zur Verfügung.
\begin{figure}[h]
	\centering
	\begin{tabular}{cc}
	\includegraphics[width=6cm]{img/pelican} &
	\includegraphics[width=6cm]{img/camera}
	\end{tabular}
	\caption{AscTec Pelikan (links), MatrixVision BlueFOX mv-MLC200wC (rechts)}
	\label{img:pelican}
\end{figure}


\section{Ziel der Arbeit}
\label{sec:ziel_der_arbeit}
Basierend auf photogrammetrischen Konzepten ist es möglich räumliche Tiefe aus zweidimensionalen Bilddaten zu berechnen. Generell werden mindestens zwei Bilder aus unterschiedlichen Standpunkten für die Berechnung dreidimensionaler Informationen benötigt. Das daraus resultierende Problem welches bei der Benutzung einer Kamera entsteht ist die oft ungenau Schätzung der relativen Orientierung sowie die damit einher gehende Skalierung. Bei der Verwendung eines Stereosystems ist die relative Orientierung beider Kameras zueinander bereits bekannt, was eine direkte Berechnung metrischer Koordinaten ermöglicht. Da Zuverlässigkeit sowie Genauigkeit vor allem im Kontext der Navigation in Innenräumen sehr wichtig ist wurde in dieser Arbeit auf ein Stereo-Setup gesetzt.\\

\noindent
Vor diesem Hintergrund werden in Kapitel \ref{chp:concepts} der Arbeit zugrundeliegende Algorithmen und Konzepte erläutert. Weiterhin wird das entwickelte Framework zur Bildaufnahme und Vorprozessierung der Bilder grundlegend beschrieben. Anschliessend werden einige State of the Art Methoden der Hinderniserkennung beschrieben wobei dabei zwischen aktiven und passiven optischen Systemen unterschieden wird. Diese Einteilung dient einerseits dafür einen Überblick über bereits bestehende Techniken sowie implementierte Systeme zu erhalten, andererseits um auch die Vor- und Nachteile der jeweiligen Technik herauszuarbeiten. Im Anschluss daran beschreibt Kapitel \ref{chp:developed_algorithms} die beiden entwickelten Methoden zur bildbasierten Hinderniserkennung und deren Implementierung detailliert. Anschließend erfolgt die Evaluation beider Verfahren wobei die Erkennung verschiedener Hindernisgrößen getestet wird. Weitere Tests zur zukünftigen Verbesserung der Algorithmen weisen auf welches aktuellen Limitierungen durch die verwendeten Konzepte vorliegen. Kapitel \ref{chp:conflicts} erläutert eben diese Limitierungen und gibt Ansätze zur Lösung aus der Fachliteratur sowie eigene Konzepte zur Bewältigung dieser. Die anschliessende Diskussion wertet die im Rahmen der Evaluation in Kapitel \ref{chp:evaluation} erlangten Ergebnisse weitergehend aus und stellt beide Algorithmen hinsichtlich der Robustheit der Erkennung, sowie der erreichten Performance aus. Kapitel \ref{chp:fazit} zieht ein Ré­su­mé aus den Ergebnissen der Arbeit und gibt einen Ausblick auf mögliche zukünftige Arbeiten in diesem Bereich.

%------------------------------------------------

\chapter{Zugrunde liegende Konzepte und Algorithmen}
\label{chp:concepts}
% ---------------------- section -----------------------
\section{Epipolargeometrie}
\label{sec:epipolargeometrie}


% ---------------------- section -----------------------
\section{Stereo Matching}
\label{sec:stereo_matching}

\subsection{Klassifikation}
\label{subsec:stereo_matching_classification}

\subsection{Block Matching}
\label{subsec:stereo_matching_bm}

\subsection{Semi Global Block Matching}
\label{subsec:stereo_matching_sgbm}


% ---------------------- section -----------------------
\section{mvStereoVision Framework}
\label{sec:framework}


%------------------------------------------------

\chapter{Optische Verfahren zur Hinderniserkennung}
\label{chp:stateoftheart}
% ---------------------- section -----------------------
\section{Kamerabasierte Hinderniserkennung}
\label{sec:kamera_basierte_he}

% ---------------------- section -----------------------
\section{Sensorbasierte Hinderniserkennung}
\label{sec:sensor_basierte_he}


%------------------------------------------------

\chapter{Entwickelte Hinderniserkennungen}
\label{chp:developed_algorithms}
Im Rahmen dieser Arbeit wurden zwei Systeme zur Erkennung von Hindernissen in Echtzeit entwickelt. Diese richten sich nach den in Kapitel REF erläuterten Algorithmen und Konzepten. Anhand dieser ist es möglich aus den beiden Bildern des Stereo Systems für jeden Frame die Disparity Map zu berechnen, welche im Anschluss daran in mehreren Schritten zunächst so angepasst wird, dass nicht verwertbare Bereiche der Tiefenkarte entfernt werden (siehe REF preprocessing). Dies reduziert die Anzahl der zur Erkennung zu verarbeitenden Punkte und somit die Rechenzeit.\\

\noindent
Im folgenden Kapitel werden beide Methoden detailliert beleuchtet. Zu Beginn wird die zugrunde liegende Klassenstruktur beschrieben. In Abschnitt \ref{sec:mean_disparity_detection} die \emph{Mean Disparity Detection} erläutert, wobei auf das grundlegende Konzept sowie den Algorithmus zur Erkennung selber eingegangen wird. Selbiges gilt für die \emph{Samplepoint Detection} in Abschnitt \ref{sec:samplepoint_detection}.


\begin{figure}[h]
	\begin{center}
	    %TODO: change Samplepoint Detection mImageCenter to private
	    %TODO: add Subimage Stuff to MeanDisparityDetection
		\includegraphics[width=13cm]{img/obstacle_detection_structure.pdf}
	\end{center}
	\caption{Klassenstruktur der Hinderniserkennung}
	\label{fig:obstacle_detection_structure}
\end{figure}



% ---------------------- section -----------------------
\section{Mean Disparity Detection}
\label{sec:mean_disparity_detection}

\begin{algorithm}[h]
\caption{Berechnung des Disparity Medians}
\label{alg:mean_disparity_calculation}
\begin{algorithmic}[1]
    \Procedure{CalcMeanDisparity}{$submatrix$}
        \State $elements_{number} \gets 0$
        \State $elements_{sum} \gets 0 $
        \For{$value$ in $submatrix$}
            \If{$value > 0$}
                \State $elements_{sum} \gets elements_{sum} + value$
                \State $elements_{number} \gets elements_{number} + 1$
            \EndIf
        \EndFor
        \State \textbf{return} $elements_{sum} / elements_{number}$
    \EndProcedure
\end{algorithmic}  
\end{algorithm}

% ---------------------- section -----------------------
\section{Samplepoint Detection}
\label{sec:samplepoint_detection}



%------------------------------------------------

\chapter{Evaluation}
\label{chp:evaluation}
Um die Funktionsweise beider Algorithmen zu belegen wurden beide Methoden einerseits auf ihre Robsutheit und andrerseites auf ihre Performance getestet. Im folgenden Kapitel wird zunächst das zum Testen verwendete Setup beschrieben. Anschliessend erfolgt die Auswertung der erlangten Testergebnisse. Zuletzt werden weitere Tests durchgeführt welche das System kritischen Situationen testen soll.

% ---------------------- section -----------------------
\section{Testsetup}
\label{sec:aufbau}

Bei der Durchführung der Tests befinden sich die Kameras statisch im Raum. Die zu erkennenden Hindernisse werden innerhalb und ausserhalb der zu erkennenden Reichweite platziert, wobei die Ausrichtung der Hindernisse teils zufällig, teils bewusst an kritischen Positionen erfolgt, um ein reales Anwendungsszenarien passend zu simulieren. Ein aufgenommenes Testset besteht dabei aus beiden Bildern der Kamera, der normalisierten Disparity Map sowie eine komplette Pointcloud dieser um etwaige Fehler der Algorithmen leichter erkennen zu können, sowie den geloggten Pointclouds der Hinderniserkennung. Weiterhin werden diverse Parameter gespeichert, wie die Anzahl der erkannten Hinderniselemente, sowie deren Disparitäten.\\

\noindent
Der zu erkennende Bereich wurde auf $0,2$ bis $1,5$ Meter definiert. Dies entspricht einem Szenario in welchem das System auch aufgrund der hohen Framerate der Erkennung angewendet werden kann. Eine Erweiterung dessen auf beispielsweise $2.0$ Meter wurde nicht durchgeführt, da der Algorithmus auch bei großen Entfernungen robuste Werte in der Distanzberechnung liefert \cite{hilleralhallak}.
Das dabei erreichte Sichtfeld nach der Anwendug der ROI auf die Disparity Map (siehe \ref{sec:preprocessing}) beträgt $50^{\circ}$ auf horizontaler Achse und $38^{\circ}$ vertikal. Dies ist in Abbildung (REF) visualisiert.
	% TODO sichtfeld durch disparity map & vergleich zum ursprünglichen sichtfeld

\noindent
Ein aufgenommenes Testset besteht aus jeweils 12 Testbildern. Für jede Methode wurden drei verschiedene Hindernisgrößen getestet, groß, klein und winzige Hindernisse. Anhand dieser wird ausgewertet welche minimale, maximale sowie mittlere Disparität, und daraus resultierende Distanz erkannt wird.\\

\noindent
Weitere Tests beinhalten die Erkennung winziger Hindernisse unter Veränderung der \emph{SGBM} Parameter. Dabei wird unter anderem untersucht ob beispielsweise eine verringerte Blockgröße Einfluss auf die Erkennung kleiner Bereiche nimmt. Des Weiteren wird die Zeit für die Hinderniserkennung eines Frames untersucht um eine durchschnittliche Zeit für die Erkennung sowie die daraus resultierende Framerate zu ermitteln. Dies geschieht einerseits durch die Erkennung eines Hindernisses, welches sich über das gesamte Bild ausbreitet, andererseits für nur ein Teilelement jeder Erkennungsmethode (Subimage, Samplepoint).\\
Zudem wird geprüft inwiefern die Algorithmen mit Limitierungen des \emph{SGBM} umgehen können. Dazu zählen die Erkennung bei spiegelnden, reflektierenden und durchsichtigen Flächen, sowie die Erkennung schwach texturierter Hindernisse.\\

% kameras statisch im raum
% hindernisse innerhalb der zu erkennenden reichweite platziert
% ausrichtung der hinernisse zum teil zufällig
% teils bewusst an kritischen positionen platizert
% für jeden Frame werden Informationen geloggt wie anzahl der erkannten hindernisse sowie position der hindernisse im raum
% für jeden frame wird eine vollständige pointcloud berechnet um etwaige fehler zu erkennen
% verschiedene hindernisgrößen gross klein winzig
% jeweils 12 testbilder
% range auf 0.2 - 1.5 meter festgelegt --> real life szenario

% weitere tests:
	% erkennung von kleinen hindernissen mithilfe veränderter sgbm parameter
	% zeit die fuer die ausführung der detectObstacles funktion benötigt wird
	% können auch extreme situationen wie durchsichtige flächen erkannt werden?
		% spiegelnd
	% erkennung von wenig texturierten hindernissen
	% 

% ---------------------- section -----------------------
\section{Evaluierung Subimage Detection}
\label{sec:evaluierung_subimage}

    \subsection{Robustheit}
    \label{subsec:subimage_robustheit}
    
    % allgemein hindernisse beschreiben
    	% welche fläche repräsentiert jedes hinderniss
    	% was ist daran gut / scchelcht vielleicht
    
	% hinsichtlich robustheit
	% beschriebene testsets werden ausgewertet
	% nach der Anzahl der erkannte hindernisse
	% den gespeicherten weltdistanzen
	% den daraus resultierentden median werten der distanz
		% berechnet und gemessen
	% werden Hindernisse erkannt
	% wie hoch ist der drift zwischen den einzelnen sets
	% vielleicht standartabweichung betrachten
	% für jedes set einzeln
	% woher kommen die drifts 
		% (person im bild bei der bildaufnahme
		% hindernisse im winkel gehalten --> geneigt
	% trotzallem wurde jedes hindernis erkannt
	% abweichung entsteht einerseits durch besagten drift durch beschriebene gruende
		    
\begin{figure}[h]
	\centering
	\begin{tabular}{m{7.0cm} m{7.0cm}}
	\includegraphics[width=7cm]{img/evaluation/big_bar}
	\centering \small (a)
	&
	\includegraphics[width=7cm]{img/evaluation/big_box}
	\centering \small (b)
	\end{tabular}
    \caption{}
    \label{fig:eval_big}
\end{figure}

\begin{figure}[h]
	\centering
	\begin{tabular}{m{7.0cm} m{7.0cm}}
	\includegraphics[width=7cm]{img/evaluation/medium_bar}
	\centering \small (a)
	&
	\includegraphics[width=7cm]{img/evaluation/medium_box}
	\centering \small (b)
	\end{tabular}
    \caption{}
    \label{fig:eval_medium}
\end{figure}

\begin{figure}[h]
	\centering
	\begin{tabular}{m{7.0cm} m{7.0cm}}
	\includegraphics[width=7cm]{img/evaluation/tiny_bar}
	\centering \small (a)
	&
	\includegraphics[width=7cm]{img/evaluation/tiny_box}
	\centering \small (b)
	\end{tabular}
    \caption{}
    \label{fig:eval_tiny}
\end{figure}

    \subsection{Performanz}
    \label{subsec:subimage_performanz}

% ---------------------- section -----------------------
\section{Evaluierung Samplepoint Detection}
\label{sec:evaluierung_samplepoint}

    \subsection{Robustheit}
    \label{subsec:samplepoint_robustheit}    

    \subsection{Performanz}
    \label{subsec:samplepoint_performanz}

%------------------------------------------------

\chapter{Limitierungen und Lösungsansätze}
\label{chp:conflicts}
In der Erkennung von Hindernissen mithilfe von passiv optischen Systemen können verschiedenste Faktoren der Grund für eine fehlerhafte Erkennung sein. Sei es die Berechnung einer Disparity Map von Bereichen mit einer Vielzahl homogener oder reflektierender Flächen oder die Veränderung der Lichtverhältnisse in einem der beiden Kamerabilder. In diesem Kapitel werden einige der bestehenden, sowie einige potentiell mögliche Fehlerquellen erläutert sowie dazugehörige Lösungsansätze entwickelt.\\

Abschnitt \ref{sec:obstacle_validation} geht dabei auf eine Möglichkeit zur Validierung der Hindernisse ein, welche auch in den entwickelten Algorithmen implementiert ist. Darauf folgend werden in Abschnitt \ref{sec:existing_conflicts} noch bestehende Konflikte in der Erkennung dargestellt sowie Lösungsansätze für diese gegeben.

% ---------------------- section -----------------------
\section{Validierung der Hindernisse}
\label{sec:obstacle_validation}

Um die Wa

% ---------------------- section -----------------------
\section{Bestehende Konflikte und Lösungsansätze}
\label{sec:existing_conflicts}




%------------------------------------------------

%\chapter{Diskussion}
%\label{chp:discussion}
%Im Rahmen des Kapitels \enquote{Evaluation} wurden beide entwickelten Methoden in verschiedenen Versuchen getestet. Die erlangten Ergebnisse beider Algorithmen unterscheiden sich in diversen Bereichen. Es erfolg somit eine Gegenüberstellung beider Algorithmen in Hinsicht auf die Robustheit der Erkennung sowie deren Performance.

% generell ist zu sagen das die mean besser funktioniert wenn die hindernisse nicht vor einem weit entfernten hintergrund platziert werden

% durch die geringen disparitäten der hinteren 'ebene' wirrd der median so weit verzerrt, dass kleine objekte einfach nicht mehr auffallen

% wenn hintergrund höhere disparitäten aufweist fallen hohe disparitäten auch mehr ins gewicht

% zum teil auch so bei der samplepoint detection, jedoch erwies sich diese in den tests als robuster

% gerade kleine Hindernisse wurden auf distanz besser erkannt

% trotzallem nicht immer eindeutig teilweise wurde ein Teil des objektes erkannt wie der stab aber nicht das eigentlich gesuchte objekt

% 

% ---------------------- section -----------------------
\section{Gegenüberstellung beider Algorithmen}
\label{sec:gegenueberstellung}

\subsection{Performance}
\label{subsec:discussion_performance}

Hinsichtlich der Performance der entwickelten Systeme wurden verschiednen Punkte betrachtet. Die zugrunde liegende Berechnung der Disparity Map ist der wohl wichtigste Faktor. Ist dieser Prozess langsam, so ist auch die Performance der Hinderniserkennung eingeschränkt. Weiterhin wird auch die Performance der einzelnen Methoden untersucht. Dabei werden folgende Situationen untersucht:
\begin{enumerate}
	\item Analyse der gesamten Schleife des Hauptprogramms
	\begin{enumerate}
		\item Hindernisse bewegen sich durch die Gefahrenzone
		\item der gesamte Sichtbereich ist mit einem Hindernis gefüllt
		\item es befindet sich kein Hindernis in der Gefahrenzone
	\end{enumerate}
	\item Analyse einzelner Erkennungssschritte
	\begin{enumerate}
		\item Geschwindigkeit der Update Funktion
		\item Geschwindigkeit der detectObstacles Funktion ohne Hindernisse
		\item Geschwindigkeit der detectObstacles Funktion mit einem Hindernis im gesamten Sichtbereich
	\end{enumerate}
\end{enumerate}

\noindent
Zu Beginn ist ein essenzieller Schritt die Geschwindigkeit der Disparity Map Berechnung zu analysieren. Dabei wurden beide Kameras im gebinnten Modus getestet. Die Aufnahmerate der Kameras beträgt dabei 50 Frames pro Sekunde bei einer Verschlusszeit von 10000 µs. Die unter Veränderung der Blockgröße erhaltenen Parameter sind in Tabelle \ref{tbl:disparity_framerate} dargestellt. Aus dieser ist zu erkennen, dass eine Modifikation dieses Parameters nur marginale Änderungen in der Geschwindigkeit auftreten.

\begin{table}[h]
	\centering
	\begin{tabular}{|l|c|c|}
	\hline
	Block Größe & Zeit pro Frame & Frames pro Sekunde \\ \hline
	7           & 0.0412         & 24.21            \\ \hline
	9           & 0.0420         & 23.77            \\ \hline
	11          & 0.0413         & 24.17            \\ \hline
	13          & 0.0410         & 24.36            \\ \hline
	15          & 0.0412         & 24.22            \\ \hline
	21          & 0.0413         & 24.17            \\ \hline
	\end{tabular}
	\caption{Blockgröße und daraus resultierende Frameraten}
	\label{tbl:disparity_framerate}
\end{table}

\noindent
Die dabei gemessene Bildwiederholrate von 24 Einzelbildern pro Sekunde ist eine gute Voraussetzung für die Hinderniserkennung. Unter der Annahme das es zu keiner Verlangsamung dieser kommt ist es möglich sich mit Einer Geschwindigkeit von $12\frac{m}{s}$ zu bewegen und für jeden zurückgelegten Meter 2 berechnete Disparity Maps zu erhalten. Eine solche Geschwindigkeit ist bei besagter Framerate nicht die präferierte Geschwindigkeit jedoch potentiell möglich. Zudem in weiteren Schritten mehr Zeit für die Entwicklung einer Vermeidungsstrategie in Betracht gezogen werden muss.\\

\noindent
Die Geschwindigkeit der eigentlichen Hinderniserkennung ist ebenfalls ein wesentlicher Faktor in der Betrachtung der gesamten Performance. Dazu wurden besaget Tests durchgeführt. Die daraus erhaltenen Ergebnisse für die Subimage Detection finden sich in Tabelle \ref{tbl:subimage_framerate}.

%TODO FIX THIRD ROW 
\begin{table}[h]
\centering
\begin{tabular}{|l|c|c|}
\hline
Szenario & Zeit pro Frame (Detection) & Detection fps \\ \hline\hline
1(a)     & 0.0057                     & 173.35        \\ \hline
1(b)     & 0.0067                     & 147.69        \\ \hline
1(c)     & 0.0067                     & 147.69        \\ \hline\hline
2(a)     & 0.0021                     & 469.93        \\ \hline
2(b)     & 0.0001                     & 8759.63       \\ \hline
2(c)     & 0.0042                     & 234.64        \\ \hline
\end{tabular}
\caption{Gemessene Einzelbilder pro Sekunde sowie Gesamtframerate}
\label{tbl:subimage_framerate}
\end{table}

\noindent
Aus dieser wird ersichtlich, dass Szenario 1(a), welches einer echten Anwendung am nächsten kommt, bereits eine Framerate von 173 Einzelbildern pro Sekunde aufweist. Die darauf folgendenTests bestätigen die Annahme, dass keine wesentlich langsamere Framerate aufgrund der Hinderniserkennung zu erwarten ist. Im schlechtesten Fall, einem Hindernis welches den gesamtem Sichtbereich einnimmt ist die kombinierte Framerate nicht geringer als XXX wie die folgende Rechnung aufzeigt. Dabei entsprechen die genutzten Werte denen aus \ref{tbl:disparity_framerate} mit 13 Pixeln Blockgröße sowie \ref{tbl:subimage_framerate} 1(b). 

\begin{equation}
\label{eq:fps_calculation}
\begin{aligned}
	fps &= \frac{1}{t_{frame}}\\
	t_{frame} &= 0.0410 + 0.0067 = 0.0477\\
	fps &= \frac{1}{0.0477} = 20,96
\end{aligned}
\end{equation}

\noindent


\begin{table}[h]
\centering
\begin{tabular}{|l|l|l|}
\hline
Szenario & Zeit pro Frame (Detection) & Detection fps \\ \hline\hline
1(a)     & 0.0074           			  & 133.61         \\ \hline
1(b)     & 0.0100           			  & 99.33          \\ \hline
1(c)     & 0.0016           			  & 606.84         \\ \hline\hline
2(a)     & 0.0015           			  & 647.07         \\ \hline
2(b)     & 0.0015           		  	  & 647.07         \\ \hline
2(c)     & 0.0083           	 		  & 120.33         \\ \hline
\end{tabular}
\caption{Gemessene Einzelbilder pro Sekunde sowie Gesamtframerate}
\label{tbl:samplepoint_framerate}
\end{table}

% hinsichtlich performance verschiedene Punkte betrachtet
% erkennung pro frame im mittel
% framerate der Disparity berechnung
% gesamte framerate und ausblick auf die darurch erreichbare performance
% evtl auslagern und eigenne section machen da beide direkt gegenübergestllt werden können.
    
% resizing der initalbilder zur performance verbesserung

\subsection{Robustheit}
\label{subsec:discussion_robustness}
Wie die Evaluation bereits aufzeigt ist die Erkennung unterschiedlicher Hindernisgrößen als robust anzusehen. Beide Algorithmen erkennen sowohl große als auch kleine Hindernisse innerhalb der definierten Gefahrenzone. Die ermittelten Distanzen weisen zwar kleine Ungenauigkeiten auf, jedoch sind diese eher ein Resultat von Messungenauigkeiten sowie der verwendeten Bildgröße. Auch die dahingehend erstellten Punktwolken liefern genaue Positionsinformationen ausgehend von der aktuellen Weltposition der Drohne. Jene liegen im Rahmen dieser Arbeit nicht vor da dies als ein anderer Teil des SLAM Forschungsfeldes anzusehen ist.\\

% TODO Bilder pointcloud mean und sample desselben Objektes

\noindent
Bei der Erkennung kleiner Hindernisse ist jedoch zu erkennen, dass die entwickelte Samplepoint Detection wesentlich robustere Ergebnisse liefert als die Subimage Detection. Dies resultiert vornehmlich aus der signifikant kleineren Anzahl an Pixeln. Dadurch sind diese weniger empfänglich für Verzerrungen der berechneten Distanz wie Subimages. Enthält ein einziger Samplepoint zu wenige Daten um als Hindernis angesehen zu werden, so ist die Wahrscheinlichkeit das seine Nachbarn diese Information enthalten bzw. erfassen konnten höher als beispielsweise die benachbarten Subimages. Dies ist auch als der große Vorteil der Samplepoint Detection anzusehen.\\

\noindent
Weiterhin ließ sich während der Versuchsdurchführung deutlich erkennen, dass Bewegungen einen wesentlich Bestandteil der Hinderniserkennung darstellt. Wurden die Hindernisse bewegt, konnte gerade im Fall der Subimage Detection festgestellt werden, dass die Erkennung kleiner Hindernisse signifikant besser funktionierte wenn das Objekt Bewegung aufweist. Dadurch eliminieren sich bereits beschriebene Konfliktfälle in denen sich das zu erkennende Hindernis an der Kreuzung mehrerer Subimages befand. Die Bewegung sorgte in diesem Fall dafür das die Hindernisse in mehr Frames erkannt wurden als in einer statischen Szene. Selbiges Phänomen trat auch bei der Samplepoint Detection auf.\\

\noindent
Eine Veränderung der \emph{SGBM} Blockgröße hatte keine signifikanten Auswirkungen auf die Robustheit beider Algorithmen. Lediglich die bereits in Abschnitt \ref{sec:evaluation_Diskussion} erläuterten Ergebnisse bei 13 Pixeln brachte eine geringe Verbesserung der berechneten Distanz.\\

\noindent

% Auf eine genaue Positionierung jedes erkannten Hindernisses sowie eine Segmentierung nach erkannten Bereichen wurde bewusst verzichtet, da die berechneten dreidimensionalen Koordinaten als solche für eine weitere Hindernisvermeidung ausreichen.
% bisschen mehr statistik kram
% welcher ist wo signifikant besser
	% bezug auf hindernisgroesse
	% wahrscheinlichkeit das hindernis schlecht uz erkennen aber trotzdem wird es erkannt

% bei statischen szenen schlechtere erkennung, bewegung hilft jedoch hindernisse zu erkennen, da diese durch die in der evaluierung angesprochenen bereiche (kreuzungen, nicht betrachteter bereich) durchwanderen und somit eine erkennung auslösen.
 
% bewegung fundamental wichtig





%------------------------------------------------

\chapter{Fazit und zukünftige Arbeiten}
\label{chp:fazit}
Lorim ipsum dolor sit amet.



% performance sehr gut
% nicht parallelisiert --> einerseits genug raum fuer SLAM algorithmen oä, andererseits weniger performance in der hinderniserkennung jedoch gut daher wahrscheinlich egal
% tests des aktuellen systems in während echter fluege, auswertung der daten frame by frame
% erkennung kleiner hindernissc im binning mode muss verbessert werden oder performance des ungebinnten höher
% implementierung in ros node weiterleitung der ergebnisse mithilfe von messages kp wie das geht
% limitierungen klären
% eigener disparity algorithmus welcher evll nur fuer diesen gebrauch ausgelegt ist

%------------------------------------------------
%	APPENDIX
%------------------------------------------------

\appendix

%------------------------------------------------
%	IMAGES APPENDIX
%------------------------------------------------

%\listoffigures

%------------------------------------------------
%	TABLES APPENDIX
%------------------------------------------------

%\listoftables

%------------------------------------------------
%	REFERENCES
%------------------------------------------------

\bibliographystyle{apalike}    
\bibliography{main_bib.bib}

% book sources
\nocite{zureiki2008stereo}
\nocite{opencvoreilly}
\nocite{cyganek2011introduction}

%------------------------------------------------
%	END OF DOCUMENT
%------------------------------------------------

\end{document}
