In der Erkennung von Hindernissen mithilfe von passiv optischen Systemen können verschiedenste Faktoren der Grund für eine fehlerhafte Erkennung sein. Sei es die Berechnung einer Disparity Map von Bereichen mit einer Vielzahl homogener oder reflektierender Flächen oder die Veränderung der Lichtverhältnisse in einem der beiden Kamerabilder. In diesem Kapitel werden einige der bestehenden, sowie einige potentiell mögliche Fehlerquellen erläutert sowie dazugehörige Lösungsansätze entwickelt.\\

\noindent
Abschnitt \ref{sec:obstacle_validation} geht dabei auf eine Möglichkeit zur Validierung der Hindernisse ein, welche auch in den entwickelten Algorithmen implementiert ist. Darauf folgend werden in Abschnitt \ref{sec:existing_conflicts} noch bestehende Konflikte in der Erkennung dargestellt sowie Lösungsansätze für diese gegeben.

% ---------------------- section -----------------------
\section{Bestehende Konflikte und Lösungsansätze}
\label{sec:existing_conflicts}

Die Validierung der erkannten Hindernisse ist ein kompliziertes Problem in der autonomen Hinderniserkennung. Durch etwaige äußere Einflüsse wie die Veränderung der Lichtverhältnisse kann die Berechnung der Tiefenkarte fehlerhaft sein. Dies kann im Fall eines autonomen Flugs dazu führen, dass das UAV ein Hindernis innerhalb eines Korridors erkennt und aufgrund dessen versucht diesem imaginären Hindernis auszuweichen. Gerade in engen Umgebungen ohne viel verfügbaren Platz kann dies zum totalen Systemausfall führen. Daher sollte ausgewertet werden ob es sich bei erkannten Hindernissen auch um solche handelt. Ein Ansatz zur Vermeidung solcher Falscherkennungen wäre, die Objekte insofern zu verfolgen, dass für jeden Frame (in Abhängigkeit der zugrundliegenden Framerate) überprüft wird ob bereits im vorherigen Frame ein Objekt gefunden wurde. Erst nachdem dies sichergestellt wurde wird daraufhin eine Warnung ausgegeben. Zeitgleich wäre es möglich, dass Informationen wie die Eigenbewegung sowie die Information ob sich das erkannte Objekt selber im Raum bewegt verloren gehen.

Homogene & spiegelnde Flächen:


Hindernisgrösse:



% ---------------------- section -----------------------
\section{Diskussion}
\label{sec:conflict_discussion}
