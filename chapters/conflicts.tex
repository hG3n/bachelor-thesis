In der Erkennung von Hindernissen mithilfe von passiv optischen Systemen können verschiedenste Faktoren der Grund für eine fehlerhafte Erkennung sein. Sei es die Berechnung einer Disparity Map von Bereichen mit einer Vielzahl homogener oder reflektierender Flächen oder die Veränderung der Lichtverhältnisse in einem der beiden Kamerabilder. In diesem Kapitel werden einige der bestehenden, sowie einige potentiell mögliche Fehlerquellen erläutert sowie dazugehörige Lösungsansätze entwickelt.\\

Abschnitt \ref{sec:obstacle_validation} geht dabei auf eine Möglichkeit zur Validierung der Hindernisse ein, welche auch in den entwickelten Algorithmen implementiert ist. Darauf folgend werden in Abschnitt \ref{sec:existing_conflicts} noch bestehende Konflikte in der Erkennung dargestellt sowie Lösungsansätze für diese gegeben.

% ---------------------- section -----------------------
\section{Validierung der Hindernisse}
\label{sec:obstacle_validation}

Um die Wa

% ---------------------- section -----------------------
\section{Bestehende Konflikte und Lösungsansätze}
\label{sec:existing_conflicts}


