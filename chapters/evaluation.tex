% ---------------------- section -----------------------
\section{Aufbau des Testsetups}
\label{sec:aufbau}
Ei


% kameras statisch im raum
% hindernisse innerhalb der zu erkennenden reichweite platziert
% ausrichtung der hinernisse zum teil zufällig
% teils bewusst an kritischen positionen platizert
% für jeden Frame werden Informationen geloggt wie anzahl der erkannten hindernisse sowie position der hindernisse im raum
% für jeden frame wird eine vollständige pointcloud berechnet um etwaige fehler zu erkennen
% verschiedene hindernisgrößen gross klein winzig
% jeweils 12 testbilder
% range auf 0.2 - 1.5 meter festgelegt --> real life szenario

% weitere tests:
	% erkennung von kleinen hindernissen mithilfe veränderter sgbm parameter
	% zeit die fuer die ausführung der detectObstacles funktion benötigt wird
	% können auch extreme situationen wie durchsichtige flächen erkannt werden?
		% spiegelnd
	% erkennung von wenig texturierten hindernissen
	% 

% weiterhin:
	% aktuelles sichtfeld
	% 

% ---------------------- section -----------------------
\section{Evaluierung Subimage Detection}
\label{sec:evaluierung_subimage}

    \subsection{Robustheit}
    \label{subsec:subimage_robustheit}

    \subsection{Performanz}
    \label{subsec:subimage_performanz}

% ---------------------- section -----------------------
\section{Evaluierung Samplepoint Detection}
\label{sec:evaluierung_samplepoint}

    \subsection{Robustheit}
    \label{subsec:samplepoint_robustheit}    

    \subsection{Performanz}
    \label{subsec:samplepoint_performanz}