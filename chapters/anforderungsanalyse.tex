Zur Erreichung der in Abschnitt \ref{sec:ziel_der_arbeit} beschriebenen Ziele wurden einige Anforderungen definiert. Jene Eckpfeiner der Entwicklung sollen in diesem Kapitel näher beleuchtet werden.\\

\noindent
Abschnitt \ref{req:bildaufnahme} geht hierbei auf die notwendigen Schritte des Preprocessings der aufgenommenen Bilder ein. Des Weiteren wird in Abschnitt \ref{req:performanz} die Notwendigkeit einer Erkennung der Hindernisse in Echtzeit sowie eine daraus resultierende signifikant hohe Wahrscheinlichkeit in der korrekten Erkennung der Hindernisse diskutiert. Zuletzt wird in Abschnitt \ref{req:positionsinformationen} die weitergehende Arbeit mit den errechneten Positionsinformationen behandelt.

% ---------------------- section -----------------------
\section{Bildaufnahme und Preprocessing}
\label{sec:bildaufnahme_preprocessing}

	\begin{anforderung}
	\label{req:bildaufnahme}
		Die zur Erkennung von Hindernissen verwendeten Bilder müssen kalibriert, entzerrt sowie stereo-rektifiziert vorliegen.
	\end{anforderung}

\noindent
Um Anforderung \ref{req:bildaufnahme} zu erfüllen müssen zunächst die intrinsischen und extrinsischen Parameter der Kamera berechnet werden. Dieser Prozess wird als Kalibierung der Kamera bezeichnet. Dafür werden Bilder eines Kalibierobjektes aufgenommen. Diese sind meistens einfache Schachbrettmuster mit ungleicher Anzahl an Quadraten, oder auch radiale Objekte mit einem spezifischerem Aufbau. Der Prozess der Kalibierung zielt darauf ab die 'inneren' Parameter der Kamera zu bestimmen. Dies beinhaltet einerseits die Kameramatrix in welcher Parameter wie der Fokus Punkt(Focal Point), der Bildhauptpunkt(Principal Point) sowie die Brennweite der der Kamera. Weiterhin werden geometrische Verzerrungen des Bildes,  parametrisiert und somit nutzbar für die Entzerrung der Bilder in weiteren Schritten. Weiterhin werden bei der Kallibrierung die extrinsischen (äußeren) Parameter der Kamera bestimmt. Zu diesen zählt die Translation (x,y und z) sowie die Rotation der Kamera um die jeweiligen Achsen des Weltkoordinatensystems. Ein spezieller Aspekt bei der Kalibierung von Stereo-Systemen ist die Berechnung der extrinsischen Parameter von einer Kamera zur anderen. Die dabei berechnete Translation einer Kamera zur Referenzkamera wird als Baseline bezeichnet. Diese ist für Prozeduren wie Stereo-Matching unabdingbar. Der Prozess der Parameterfindung ist in der Regel ein Prozess welcher einmalig nach dem Aufbau ausgeführt werden muss und solange währt wie sich nichts am Aufbau der Kameras ändert (Objektive, Translation, Rotation).\\

\noindent
Folgend aus der Kalibrierung der Kameras beginnt der Prozess der Rektifizierung der Bilder. Dieser kann dabei in zwei Schritte unterteilt werden, einerseits die Berechung der Region of Interest, andererseits die Rektifizierung jedes aufgenommenen Einzelbildes. Mithilfe der zuvor berechneten intrinsischen und extrinsischen Parameter kann nun die Transformation der beiden Kamerabilder zueinander berechnet werden. Dies geschieht unter dem Aspekt der Epipolargeometrie. Nach der Rektifizierung und dem Anwenden der ROIs auf beiden Kamerabildern entspricht die jeweilige Pixelreihe im linken Bild der Pixelreihe mit dem selben Index im rechten Bild.\\
    % TODO retification beschreiben

% ---------------------- section -----------------------
\section{Performanz}
\label{sec:performanz}

	\begin{anforderung}
	\label{req:performanz}
		Das System sollte Hindernisse in Echtzeit erfassen und erkennen können um eine signifikant hohe Erkennungswahrscheinlichkeit zu gewährleisten.
	\end{anforderung}

\noindent
Die Erfassung von Hindernissen in Echtzeit ist ein wesentlicher Bereich in der Entwicklung autonomer Systeme um Unfälle jedweder zu vermeiden. Dabei ist die erste Vorraussetzung für das System eine hohe Framerate in der Bildaufnahme um ausreichend Einzelbilder für die Berechnung der Disparity Map bereitzustellen. Die Anzahl der pro Sekunde verfügbaren Tiefenkarten ist ebenfalls entscheidend für die Validierung der zu erkennenden Hindernisse sowie die Geschwindigkeit in der sich das UAV letztenendes fortbewegen kann ohne aufgrund fehlender Einzelbilder Hindernisse zu verpassen.
Da die Progresse der Einzelbilder die Geschwindigkeit der Berechnung der Tiefenkarte aktiv beeinflusst muss entschieden werden ob ein erhöhter Detailgrad oder die Maximierung der Framerate von höherer Priorität sind.
% TODO mehr zu hinderniserkennung in echtzeit
	

% ---------------------- section -----------------------
\section{Positionsinformationen}
\label{sec:positionsinformationen}

	\begin{anforderung}
	\label{req:positionsinformationen}
		Die relativen Positionen der gefundenen Objekte müssen in jedem Frame vorliegen.
	\end{anforderung}
	
In Hinblick auf die Vermeidung der erkannten Hindernisse wurde Anforderung \ref{req:positionsinformationen} definiert. Dabei genügt es nicht dem System mitzuteilen das ein gefährliches Objekt gefunden wurde, da die Position für die Vermeidung dessen von Nöten ist. In dem von Richards et al. \cite{richards2014obstacle} entwickelten Algorithmus liegen die Weltkoordinaten als Resultat aus dem Optischen Fluss sowie dem Feature Tracking vor. Diese Berechnung für jeden Pixel einer Disparity Map vorzunehmen ist jedoch weder Zeit- noch Ressourcensparend und aufgrund dessen nicht für eine Echtzeiterkennung geeignet. Daher sollten nur die 3D-Informationen für die erkannten Hindernisse gespeichert bzw. an die Hindernisvermeidung weitergegeben werden.
% TODO mehr zu relativen positionen

