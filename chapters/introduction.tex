\section{Motivation}
\label{sec:uav}

Die Forschung im Bereich der Robotik ist ein immer weiter fortschreitender Prozess. Eine besondere Domäne innerhalb dessen ist die autonome Navigation sowie Steuerung von unbemannten Flugsystemen. Dies findet Anwendung in der Erkundung von unbekannten oder gefahrenträchtigen Gebieten. Dabei müssen nicht nur die physikalischen Eigenschaften der Drohne betrachtet werden sondern auch die Fusion verschiedenster Sensoren. \\

Ein wichtiges Kriterium in der Entwicklung autonomer Roboter ist die Erkennung von Hindernissen in Echtzeit. Dabei muss jedoch zuerst definiert werden was vom System als potentielles Hindernis erkannt werden soll. Einerseits sind alle Objekte welche sich in der unmittelbaren Nähe des Systems befinden eine Gefahrenquelle, andererseits wird im Falle Kamerabasierter Navigation von einer Bewegung in Blickrichtung ausgegangen. Dies schließt Objekte welche sich außerhalb des Sichtfeldes, also neben oder hinter der Kamera befinden, für die Erfassung durch die Algorithmen aus. Auch sehr weit entfernte Objekte sind prinzipiell nicht als Hindernis anzusehen, wobei sich die maximale Gefahrendistanz relativ zur aktuellen Geschwindigkeit verhält. Unter Betrachtung dieser Gesichtspunkte wird ein Hindernis innerhalb dieser Arbeit als ein Objekt definiert welches sich innerhalb eines definierten Distanzbereichs und innerhalb des Sichtfeldes der Kamera befindet. \\

Die hauptsächliche Anwendung des im Rahmen dieser Arbeit entwickelten Systems zielt auf die autonome Navigation von Unbemannten Flugsystemen ab. Weitere Anwendungsbereiche können sowohl komplexerer als auch einfacherer Natur sein. Prinzipiell ist es möglich die entwickelten Algorithmen im Automobil Bereich zu verwenden um beispielsweise Objekte vor oder Hinter dem Kraftfahrzeug zu erkennen und die Distanz zu berechnen. Weiterhin ist es vorstellbar grobe kartographische Höhen Klassifikationen vorzunehmen, um Bildbasiert eine Höhenkarte geographischer Areale zu erstellen. Das jeweilige Anwendungsszenario richtet sich jedoch dabei nach der verwendeten Hardware. 

\section{Setup}
\label{sec:applications}



\section{Ziel der Arbeit}
\label{sec:ziel_der_arbeit}
Die Verwendung Stereobild basierter Daten ermöglicht eine simple Berechnung räumlicher Tiefe.

Mithilfe photogrammetrischer Verfahren ist es mögliche räumliche Tiefe aus zweidimensionalen Daten zu berechnen. Dabei gilt die Verwendung Stereobild basierter Daten als ein einfachste Möglichkeit Disparitäten zwischen korrespondierenden Pixeln zu bestimmen um damit einen räumlichen Eindruck der abgebildeten Szenerie zu erhalten. Unter Betrachtung dieser Techniken wurde im Rahmen dieser Arbeit 

Vor diesem Hintergrund werden diverse Anforderungen definiert, welche in Kapitel 2 näher beschrieben werden. Die zugrunde liegende Berechnung von \emph{Disparity Maps} basiert auf dem Semi Global Block Matching Algorithmus [\cite{sgbm}] sowie dem Konzept der Epipolargeometrie welche in Kapitel 4 näher erläutert werden. Kapitel 3 gibt einen Einblick in State of the Art Algorithmen zur Hinderniserkennung. Dabei wird sowohl auf Kamerabasierte als auch auf Sensorbasierte Algorithmen eingegangen. In den Kapiteln 5 und 6 werden die beiden im Rahmen dieser Arbeit entwickelten Algorithmen zur Hinderniserkennung vorgestellt. Um die Echtheit gefundener Hindernisse zu validieren wurde das Verfahren \emph{Frame Skipping} implementiert welches in Kapitel 7 erläutert wird. Ebenfalls werden bestehende Konflikte in der Erkennung sowie praktische Lösungen als auch hypothetische Ansätze zur Bewältigung dieser erläutert. Die Evaluierung der beiden implementierten Methoden erfolgt in Kapitel 8. Hierbei werden diese hinsichtlich ihrer Robustheit und Performanz statistisch ausgewertet. Kapitel 9 diskutiert die Erfüllung der durch die Anforderungsanalyse definierten Vorgaben. Des Weiteren erfolgt eine Gegenüberstellung der beiden entwickelten Algorithmen. Abschließend fasst Kapitel 10 die Beiträge dieser Abschlussarbeit zusammen und gibt einen Ausblick auf mögliche zukünftige Arbeiten in diesem Bereich.