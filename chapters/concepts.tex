Dieses Kapitel beleuchtet dieser Arbeit zugrunde liegende Konzepte und Algorithmen. Zunächst wird das Prinzip der Epipolargeometrie beschrieben welches ein wesentlicher Bestandteil photogrammetrischer Verfahren ist. Daraufhin folgt eine Erläuterung des Terms Stereo Matching sowie eine Klassifizierung in lokale und globale Algorithmen. Im Anschluss daran wird das Prinzip des Block Matching Algorithmus grundlegend beschrieben, welcher die Grundlage für den im Rahmen dieser Arbeit verwendeten Semi Global Block Matching Algorithmus bildet. Anschließend erfolgt eine Erläuterung des verwendeten Frameworks sowie Details zur Implementierung dessen.

% ---------------------- section -----------------------
\section{Epipolargeometrie}
\label{sec:epipolargeometrie}


% ---------------------- section -----------------------
\section{Stereo Matching}
\label{sec:stereo_matching}
Das grundlegende Konzept des Stereo Matchings beschreibt das Finden korrespondierender Punkte in zwei simultan aufgenommenen Bildern. Die Position der beiden Kameras ist dabei leicht versetzt, um einen jeweils anderen Blickwinkel auf die Szene zu erhalten. Mithilfe der verschiedenen Perspektiven können Disparitäten (Differenz der Projektion eines Objektes vom linken zum rechten Bild) zwischen korrespondierenden Pixeln berechnet werden. Die dabei erhaltenen Tiefeninformationen eines Objektes sind mit der errechneten Disparität sowie der relativen Position der Kameras, und deren intrinsischen Parametern REF, verbunden. Im Laufe dieses Prozesses kommt es zu zwei wesentlichen Problemstellungen, der Berechnung der Dispariät (Stereo Correspondence) sowie die Invertierung der Projektiven Geometrie um dreidimensionale Informationen aus der errechnetet Disparität zu erhalten. Sofern eine Lösung beider Probleme vorhanden ist können diese Informationen durch einfache Triangulierung errechnet werden.


\subsection{Klassifikation}
\label{subsec:stereo_matching_classification}
\textbf{Lokale Methoden:}\\
Zur Berechnung der Disparität in lokalen Stereo Matching Algorithen gilt grundlegendes Prinzip: “Finde Pixel $P_2$ korrespondierend zu $P_1$ im Referenzbild”. Dabei wird die Korellation von $P_1$ und $P_2$ durch deren lokale Umgebung bestimmt. Der Referenzpunkt ist dabei das Zentrum eines Bereiches in welchem das Matching ausgeführt wird. Geläufige Methoden dafür sind Sum of Absolute Differences (SAD) (Hirschmüller 2011), Zero-mean Normalized Cross-Correlation (ZNCC) (Chen \& Medioni, 1999), (Sára, 2002), Sum of Squared Differences (SSD) (Cox et al., 1996), etc.
Jedoch können aufgrund fehlender Beschränkungen verzerrte Tiefenberechnungen auftreten, da benachbarte Pixel verschiedene Disparitäten aufweisen können (beispielsweise an horizontalen Kanten wie Türrahmen etc.). Strukturell gesehen sind lokale Methoden einfacher gehalten als Globale Methoden was einen hohen Grad an Optimierung in der Implementierung zulässt.\\\\
\textbf{Globale Methoden:}\\
Bei globalen Stereo Matching Algorithmen wird ein globales Modell der zu betrachtenden Szene erstellt, sowie eine ebenfalls globale Kostenfunktion definiert, welche minimal gehalten werden soll. Dabei werden im Gegensatz zu lokalen Methoden Matches in einer Reihe und des gesamten Bildes verglichen. Zusätzlich zu den benachbarten Pixeln werden hier ebenfalls die Matches der angrenzenden Pixel betrachtet. Zur Vereinfachung dieses Vorgangs betrachten einige Algorithmen nur die Epipolargeometrie, wobei ein zweidimensionales Problem auf ein eindimensionales reduziert wird. Resultierend daraus liegt die Stärke globaler Methoden in der Bewältigung schwacher Texturen sowie auftretende Okklusionen und unterschiedlicher Lichteinfälle, was, aufgrund der höheren Komplexität in einem höheren Rechen- und Speicheraufand mündet. Weitere Verbesserungen der Ergebnisse können durch Techniken wie Dynamische Programmierung und Graph Cut erreicht werden.

\subsection{Block Matching}
\label{subsec:stereo_matching_bm}

\subsection{Semi Global Block Matching}
\label{subsec:stereo_matching_sgbm}


% ---------------------- section -----------------------
\section{mvStereoVision Framework}
\label{sec:framework}
