% ---------------------- section -----------------------
\section{Ergebnisse der Evlauation}
\label{sec:anforderunsevaluierung}

% generell ist zu sagen das die mean besser funktioniert wenn die hindernisse nicht vor einem weit entfernten hintergrund platziert werden
% durch die geringen disparitäten der hinteren 'ebene' wirrd der median so weit verzerrt, dass kleine objekte einfach nicht mehr auffallen
% wenn hintergrund höhere disparitäten aufweist fallen hohe disparitäten auch mehr ins gewicht

% ---------------------- section -----------------------
\section{Gegenüberstellung beider Algorithmen}
\label{sec:gegenueberstellung}

\subsection{Robustheit}
\label{subsec:discussion_robsutness}
	
	% bisschen mehr statistik kram
	% welcher ist wo signifikant besser
		% bezug auf hindernisgroesse
		% wahrscheinlichkeit das hinerniss schlecht uz erkennen aber trotzdem wird es erkannt

\subsection{Performance}
\label{subsec:discussion_performance}
  
    % hinsichtlich performance verschiedene Punkte betrachtet
    % erkennung pro frame im mittel
    % framerate der Disparity berechnung
    % gesamte framerate und ausblick auf die darurch erreichbare performance
    % evtl auslagern und eigenne section machen da beide direkt gegenübergestllt werden können.

